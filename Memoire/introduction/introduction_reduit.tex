\introduction
      \paragraph{}
      \small{
	La connaissance dans les détails du milieu dans lequel on vit ou que l'on visite 
    s'avère éssentielle.Le SIG (Sytème d'Information Géographique),véritable outil d'aide 
    qui met à disposition l'information localisée et cartographiée au grand public pourrait
    y contribuer.
	La géolocalisation mobile résultat de la combinaison du SIG,d'internet et du mobile
    est un domaine en plaine évolution ces dernières années.Cette évolution est marquée par 
    l'apparition des applications mobiles de géolocalisation telles que Google Maps,Osmand 
    et bien d'autres.Aussi la mise à disposition aux professionnels du domaine (les developpeurs surtout)
    des API (Application Programming Interface) et l'existence de nombreux outils libre ont rendu plus 
    accessible la géolocalisation mobile.Ces applications mobiles qui peuvent être de promotion touristique,
    de calcul d'itinéraire,de recherche géolocalisée de ressources d'un territoire ont ainsi fait grandir
    l'intérêt du grand public pour la cartographie mobile. 
	Le thème de notre mémoire << Système de géolocalisation sur le campus d'Abomey-Calavi >> s'inscrit dans 
    cette dynamique,Il s'agira de mettre en place une application mobile sous android de géolocalisation pour l'Université
    d' Abomey-Calavi (UAC).}
    
    \paragraph{Problématique\\ \\}
    \small{ 
	L'université d'Abomey-Calavi constitue un large et vaste territoire.Cette université contient des milliers 
     d'etudiants et acceuil chaque année un nombre important de nouveaux etudiants et chaque jour des visiteurs 
     dont les lieux d'intérets peuvent être des bâtiments administratifs,des amphitéâtres,des salles de cours,
     des restaurants,etc.La connaissance de ces lieux qui ont un intérêt pour ces usagers et visiteurs devient 
     alors un problème.En effet il est très fréquent de voir ou de rencontrer des visiteurs et même des étudiants 
     demander l'emplacement d'un lieu.Il en result une demande de géolocalisation de cette grande université.

	  De ce constat ,nous pourrons dire que << La maîtrise du milieu fréquenté quotidiennement facilite 
      et rend autonome nos déplacements>>.On n'aura pas besoin de se renseigner auprès d'une personne ,de s'égarer,
      ou de perdre son temps et même de gaspiller son carburant.

	  Et aussi pour les visiteurs <<avoir un guide de déplacement les rassure pendant leurs visites>>.

	  Ces préoccupations sont les questions aux quelles nous apporterons des réponses.\\ \\}

    \paragraph{Objectif\\ \\}
    \small{ 
	  Notre projet à pour objectif principal de mettre à la disposition des usagers de l' UAC
      un système de géolocalisation qui servira à améliorer l'autonomie et la simplicité des déplacements.
      
	  Plus précisement il s'agira de permettre aux utilisateurs de pouvoir:
	
    \begin{itemize}
	\item Se géolocaliser.
	\item Calculer un itinéraire.
	\item Rechercher un bâtiment.
	\item Consulter la fiche d’un bâtiment.
        \item Utiliser la navigation.
    \end{itemize}
    }

    \paragraph{\\}
	\small{
	  Le présent mémoire fait le point de nos travaux et comporte trois (3) chapitres.
	
	  La premier présente une revue de littérature sur la géolocalisation mobile
      sur la base de laquelle nous avons élaboré notre problématique.
    
	  Dans le deuxième chapitre, nous présentons les choix techniques opérés en
      vue de la conception et de la réalisation de la solution proposée.
    
	  Le troisième chapitre fait une analyse critique des résultats issus de nos simulations 
      après les avoir exposés.
	}


